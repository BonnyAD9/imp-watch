\documentclass{article}
\usepackage{graphicx} % Required for inserting images
\usepackage[czech]{babel}
\usepackage{hyperref}

\title{Š - ARM-KL05: LED náramkové hodinky na bázi RTC modulu}
\author{Jakub Antonín Štigler (xstigl00)}
\date{\today}

\begin{document}

\maketitle

\newpage

\tableofcontents

\newpage

\section{Popis projektu}
V projektu jsem implementoval program realizující hodinky pro mikrokontrolér
KL05Z32 s čtyř ciferným sedmi segmentovým displayem (+ tečka) za pomocí RTC
modulu. Hodinky jsou ovládány pomocí jednoho tlačítka.

Hodniky umí měřit čas a zobrazují hodiny a minuty. Dají se přepnout do
úsporného režimu kdy nesvítí display ale čas se stále počítá. Na hodinkách se
dá nastavit čas a i jas displaye.

Demonstrační video je dostupné na \href{
https://drive.google.com/file/d/11q4ZZumjkLexDSFRuGtCvZs_ZVa2TMyn/view?usp=sharing
}{google drive}.

\subsection{Použití hodinek}
Po připojení zdroje (baterky) hodinky rozsvítí display a začnou počítat čas od
půlnoci. Čas se zobrazuje pomocí 24. hodinového formátu: \verb|HH.MM|
(\verb|HH| jsou hodiny, \verb|MM| jsou minuty).

Pro ovládání hodinek slouží jedno tlačítko na kterém se rozlišuje krátké a
dlouhé stisknutí. Dlouhé stisknutí je když je tlačítko zmáčknuto na více než
1 - 2 sekundy a krátké stisknutí je když se tlačítko uvolněno dříve než uplyne
čas pro detekci dlouhého stisknutí.

Když se hodniky zapnou, tak jsou ve stavu kdy zobrazují čas. Čas je na začátku
nasavený na půlnoc (\verb|00.00|).

\subsubsection{Uspání}

Uspání hodinek se dá docílit pomocí krátkého zmáčknutí tlačítka když jsou
hodinky ve stavu zobrazování času.

Hodinky se dají opět probudit pomocí krátkého nebo dlouhého stisknutí tlačítka.

\subsubsection{Menu}

Menu se dá otevřít pomocí dlouhého zmáčknutí tlačítka když jsou hodinky ve
stavu zobrazování času. Menu obsahuje dvě možnosti:
\begin{itemize}
    \item \textbf{t} - nastavení času (time).
    \item \textbf{b} - nastavení jasu (brightness)
\end{itemize}
Označená možnost je znázorněná pomocí tečky vpravo dole vedle možnosti. Ozačená
možnost se dá vybrat pomocí krátkého zmáčknutí. Pokud je označena poslední
možnost, tak krátké zmáčknutí tlačítka zavře menu. Dlouhé podržení tlačítka
otevře příslušné nastavení (času nebo jasu) podle označené možnosti.

\subsubsection{Nastavení času}

Čas se dá nastavit otevřením možnosti \textbf{t} v menu (stisknutí ze stavu
zobrazování času: dlouhé (otevření menu), dlouhé (otevření nastavení času)).

Ve chvíli kdy se otevře nastavení času, tak se zobrazí aktuálně nastavený čas
a čas se přestane počítat. Čas se nastavuje po jednotlivých cifrách. Vybraná
cifra se dá poznat podle tečky vpravo dole od cifry.

Krátké zmáčknutí tlačítka zvíší vybranou cifru o jedna, pokud už je na
maximální hodnotě, tak se cifra nastaví zpět na 0.

Dlouhé zmáčknutí přesne výběr na další cifru. Pokud byla vabrána poslední
cifra, tak se hodinky přepnou do stavu zobrazení času a začnou zase počítat
čas.

\subsubsection{Nastavení jasu}

Jas se dá nastavit otevřením možnosti \textbf{b} v menu (stisknutí ze stavu
zobrazování času: dlouhé (otevření menu), krátké (označení \textbf{b}), dlouhé
(otevření nastavení jasu))

Hodniky podporují 11 úrovní jasu (0 - 10). Krátké zmáčknutí tlačítka zvýší jas
o 1, pokud už byl jas na maximální hodnotě, tak se nastaví zpět na 0.

Dlouhé podržení aplikuje nastavený jas a přepne hodinky zpět do stavu
zobrazování času.

\section{Způsob řešení}

Zdrojový kód jsem rozdělil do několika modulů:
\begin{itemize}
    \item \textit{main} - Ukládá stav hodinek a definuje přechody mezi stavy.
    \item \textit{clock} - Spravuje stav hodin (času).
    \item \textit{rtc} - Lehká abstrakce nad modulem RTC.
    \item \textit{gpio} - Lehká obstrakce nad modulem GPIO za cílem použití
          sedmi segmentového diplaye a tlačítka.
    \item \textit{utility} - Funkce pro aktivní čakání.
\end{itemize}

\subsection{Modul main}

Stav je ukládán ve statické globální proměnné \verb|state|. Jednotlivé stavy
jsou definovány v enumeraci \verb|State|.

Významné funkce jsou:
\begin{itemize}
    \item \verb|display| - Podle stavu rozhoduje co se má zobrazit.
    \item \verb|short_press| - Definuje chování po krátkém stisknutí tlačítka
          na základě stavu.
    \item \verb|long_press| - Definuje chování po dlouhém stisknutí tlačítka
          na základě stavu.
\end{itemize}

\subsection{Modul clock}

Čas je ukládán po jednotlivých cifrách v globálním poli \verb|time|.

Pro zvýšení času slouží funkce \verb|clock_tick|. Používá se zejména při
přerušení z RTC modulu.

\subsection{Modul rtc}

Hodiny se dají pozastavit/zapnout pomocí funkce \verb|rtc_tick|.

Funkce \verb|rtc_reset| slouží pro obnovení hodin. Používá se zejména když by
hrozilo přetečení počítadla sekund.

\subsection{Modul gpio}

Modul obsahuje enumerace pro pojmenování konfikrace displaye:
\begin{itemize}
    \item \verb|Segment| - Pojmenovává jednotlivé segmenty displaye.
    \item \verb|Digit| - Skládá segmenty displaye do znaků.
    \item \verb|Display| - Pojmenovává jednotlivé displaye.
\end{itemize}

Pro převod cifry na znak pro display slouží globální pole \verb|DIGIT|. Pro id
displaye na jeho port pak pole \verb|DISPLAY|.

Pro zobrazování na znaků display slouží funkce \verb|show|, \verb|show_msg| a
\verb|show_digs|.

Jas displaye se ovládá poměrem aktivního čekání se zapnutým displayem a s
vypnutým displayem. Toto realizuje funkce \verb|bright_wait|.

Jas se ukládá v globální proměnné \verb|brightness|.

\section{Závěr}

Hodinky umí měřit čas a podporují úsporný režim a mají jednoduché uživatelsky
přívětivé ovládání pomocí jednoho tlačítka.

Oproti zadání jde na hodinkách i navíc nastavit jas a nebylo by příliš složité
do kódu implementovat i další funkcionality, jako třeba nastavení formátu hadin
(na 12. hodinový) nebo nastavení automatického uspání/probuzení/změny jasu
podle času. Největší limitací pro implementaci je umožnit nastavení pomocí
sedmi segmentového displaye.

V implementaci jsou ošetřené i krajní případy, jako třeba přetečení počítadla
v modulu RTC. Nesem si vědom žádných problémů nebo nedostatků.

\end{document}
